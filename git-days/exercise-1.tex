%! suppress = DocumentclassNotInRoot
\documentclass[
%    ngerman,
    english,
    accentcolor=TUDa-1c,
%    dark_mode,
    fontsize= 12pt,
    a4paper,
    aspectratio=169,
    colorback=true,
    fancy_row_colors,
    boxarc=3pt,
% shell_escape = false, % Kompatibilität mit sharelatex
]{algoexercise}
\RequirePackage{import}
\RequirePackage{minted}

\subimport{../common}{packages}

\title{Git Days}
\subtitle{Day 1}

\begin{document}
    \maketitle

    \begin{task}{Git Installieren}
        \begin{enumerate}
            \item Nutzen Sie die Anleitung unter \href{https://github.com/git-guides/install-git}{https://github.com/git-guides/install-git}, um Git zu installieren.
            \item Folgen Sie den Anweisungen für Ihr Betriebssystem.
            Bei Fragen wenden Sie sich bitte an die Tutor:innen.
            \item Nach der Installation öffnen Sie ein Terminal und testen Sie die Installation mit dem Befehl \mintinline{bash}{git --version}.
            \item Konfigurieren Sie Ihren Namen und Ihre E-Mail-Adresse im globalen Kontext mit den \enquote{config}-Befehlen aus dem Vortrag.
        \end{enumerate}
    \end{task}

    \begin{task}{Baby Steps}
        \begin{subtask*}{Repository erstellen}
            Initialisieren Sie eine neue lokale Repository mittels \mintinline{bash}{git init}.
            Als initialer Commit soll ein \enquote{Hello World}-Programm erstellt werden.
            Die Verwendung von Java wird empfohlen, andere Sprachen sind jedoch ebenfalls zulässig.

            \begin{hinweis}
                Nicht vergessen, eine passende \texttt{.gitignore}-Datei für Ihre ausgewählte Sprache und Ihr Build-Tool zu erstellen!
            \end{hinweis}

        \end{subtask*}
        \begin{subtask*}{Remote Einstellen}
            Erstellen Sie eine neue Repository auf GitHub oder GitLab und setzen Sie diese als Remote mit dem Namen \enquote{origin} für Ihr lokales Repository.
            Pushen Sie Ihren initialen Commit auf den Remote.

            \begin{hinweis}
                Wir empfehlen die Verwendung von SSH (im Gegensatz zu HTTPS) für die Verbindung mit dem Remote.
            \end{hinweis}

        \end{subtask*}
    \end{task}

    \begin{task}{Calculator}
        Nun möchten Sie einen Taschenrechner entwickeln.
        Erstellen Sie dafür eine neue Klasse mit dem Namen \inlinejava{Calculator}.

        \begin{subtask*}{Basic Operations}
            Implementieren Sie die folgenden Methoden in der Klasse \inlinejava{Calculator}, sodass jede Methode in ihrem eigenen Commit festgehalten wird.
            Wenn Sie eine andere Sprache ausgewählt haben, passen Sie die Syntax an.
            \begin{itemize}
                \item \inlinejava{int add(int a, int b)}, welche zwei Zahlen addiert.
                \item \inlinejava{int sub(int a, int b)}, welche zwei Zahlen subtrahiert.
                \item \inlinejava{int mul(int a, int b)}, welche zwei Zahlen multipliziert.
                \item \inlinejava{int div(int a, int b)}, welche zwei Zahlen dividiert.
            \end{itemize}
            Pushen Sie Ihre Änderungen auf den Remote und schauen Sie sich die History im Webinterface an.
        \end{subtask*}

        \begin{subtask*}{Err: Div by zero}
            Sie haben festgestellt, dass Ihre Implementierung von \inlinejava{div} nicht mit dem Dividieren durch 0 umgehen kann.
            Schreiben Sie eine \inlinejava{if}-Anweisung, welche diesen Fall überprüft und gegebenenfalls eine \inlinejava{IllegalArgumentException} wirft.

            Erstellen Sie einen neuen Commit mit dieser Änderung und pushen Sie ihn auf den Remote.
        \end{subtask*}
    \end{task}

    \begin{task}{Fehler Finden}

        \begin{subtask*}{Repository Klonen und Branch Wechseln}
            \begin{enumerate}
                \item \mintinline{bash}{git clone git@github.com:benutzername/meinprojekt.git}
                \item \mintinline{bash}{git checkout master}
            \end{enumerate}
        \end{subtask*}

        \begin{subtask*}{Neue Änderungen Hinzufügen und Commiten}
            \begin{enumerate}
                \item \mintinline{bash}{git add .}
                \item \mintinline{bash}{git commit -m "Fügt neue Funktion hinzu"}
                \item \mintinline{bash}{git pull master}
            \end{enumerate}
        \end{subtask*}

        \begin{subtask*}{Neue Änderungen Hinzufügen und Commiten}
            \begin{enumerate}
                \item \mintinline{bash}{git commit -n "Hello World"}
                \item \mintinline{bash}{git add HelloWorld.java}
                \item \mintinline{bash}{git push master}
            \end{enumerate}
        \end{subtask*}

        \begin{subtask*}{Branch erstellen}
            \begin{enumerate}
                \item \mintinline{bash}{git branch -b feature}
            \end{enumerate}
        \end{subtask*}
    \end{task}

\end{document}
