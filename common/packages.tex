\usepackage{babel}
\usepackage{csquotes}% Anfhrungszeichen vereinfacht
\usepackage{svg}

%%%%%%%%%%%%%%%%%%%
%Paketvorschläge Tabellen
%%%%%%%%%%%%%%%%%%%
%\usepackage{array}     % Basispaket für Tabellenkonfiguration, wird von den folgenden automatisch geladen
\usepackage{tabularx}   % Tabellen, die sich automatisch der Breite anpassen
%\usepackage{longtable} % Mehrseitige Tabellen
%\usepackage{xltabular} % Mehrseitige Tabellen mit anpassbarer Breite
\usepackage{booktabs}   % Verbesserte Möglichkeiten für Tabellenlayout über horizontale Linien

%%%%%%%%%%%%%%%%%%%
%Paketvorschläge Mathematik
%%%%%%%%%%%%%%%%%%%
%\usepackage{mathtools} % erweiterte Fassung von amsmath
%\usepackage{amssymb}   % erweiterter Zeichensatz
%\usepackage{siunitx}   % Einheiten

%Formatierungen für Beispiele in diesem Dokument. Im Allgemeinen nicht notwendig!
\let\file\texttt
\let\code\texttt
\let\tbs\textbackslash
\let\pck\textsf
\let\cls\textsf

\usepackage{pifont}% Zapf-Dingbats Symbole
\newcommand*{\FeatureTrue}{\ding{52}}
\newcommand*{\FeatureFalse}{\ding{56}}

%\usepackage{standalone}
\usepackage{tikz}
\usetikzlibrary{3d, angles, animations, arrows, arrows.meta, arrows.spaced, automata, babel, backgrounds, bending, calc, calendar, chains, circuits.ee.IEC, circuits.logic.CDH, circuits.logic.IEC, circuits.logic.US, datavisualization, datavisualization.formats.functions, datavisualization.polar, decorations, decorations.footprints, decorations.fractals, decorations.markings, decorations.pathmorphing, decorations.pathreplacing, decorations.shapes, decorations.text, er, external, fadings, fit, fixedpointarithmetic, folding, fpu, graphs, graphs.standard, intersections, lindenmayersystems, math, matrix, patterns, patterns.meta, perspective, petri, plotmarks, positioning, quotes, rdf, scopes, shadings, shadows, shadows.blur, shapes, shapes.arrows, shapes.callouts, shapes.gates.logic.IEC, shapes.gates.logic.US, shapes.geometric, shapes.misc, shapes.multipart, shapes.symbols, spy, svg.path, through, tikzmark, topaths, trees, turtle, views}
\usepackage{forest}
\usepackage{fontawesome5}

\tikzset{
    tiny-commit/.style={
        circle,
        fill=TUDa-1a,
        minimum width=2.2em,
        minimum height=2.2em,
        very thick,
        draw=black!70,
    },
    tiny-merge-commit/.style={
        tiny-commit,
        fill=TUDa-1c!50,
    },
    commit/.style={
        rectangle,
        text=white,
        fill=TUDa-1c,
        very thick,
        minimum width=6em,
        minimum height=2em,
        rounded corners=0.25em,
    },
    mergecommit/.style={
        commit,
        fill=TUDa-1c!50,
    },
    ref/.style={
        commit,
        rounded corners=1em,
    },
    branch/.style={
        ref,
        minimum width=8em,
        fill=TUDa-4c
    },
    head/.style={
        ref,
        fill=TUDa-8b,
        append after command={
            (\tikzlastnode.center) node[text=white] {\texttt{HEAD}}
        }
    },
    parent/.style={-Latex, thick},
    parent-inactive/.style={parent, draw=TUDa-0b, text=TUDa-0b},
    ref-arc/.style={parent, dashed},
    node/.style={
        rectangle,
        rounded corners=1em,
        text=white,
        fill=TUDa-0d,
        very thick,
        minimum width=12em,
        minimum height=2em,
    },
    node_inactive/.style={
        node,
        fill=TUDa-0b,
    },
}
\newtcbox{\bashcommand}[1][]{
    colback=black!80,
    hbox,
    colupper=white,
    colframe=TUDa-2c!75!black,
    before upper=\textcolor{TUDa-1a}{\small\ttfamily\bfseries \$ },
    fontupper=\ttfamily,
    #1
}

\newtcbox{\bashcommandbob}[1][]{
    colback=black!80,
    hbox,
    colupper=white,
    colframe=TUDa-2c!75!black,
    before upper=\textcolor{TUDa-8a}{bob \small\ttfamily\bfseries \$ },
    fontupper=\ttfamily,
    #1
}
\newtcbox{\bashcommandalice}[1][]{
    colback=black!80,
    hbox,
    colupper=white,
    colframe=TUDa-2c!75!black,
    before upper=\textcolor{TUDa-10a}{alice \small\ttfamily\bfseries \$ },
    fontupper=\ttfamily,
    #1
}

\newcommand{\mytree}{
    \begin{forest}
        for tree={anchor=center}
        [\faFolder, s sep=1em
        [\faFolder[\faFolder][\faFolder][\faFile*]]
        [\faFolder, before computing xy={s/.average={s}{siblings}}[\faFolder][\faFile*]]
        [\faFolder[\faFile*][\faFile*][\faFile*]]
        ]
    \end{forest}
}
