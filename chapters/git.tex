\section{Git}\label{sec:git}

\begin{frame}[b]
    \begin{center}
        \includesvg[width = 8em]{../pictures/git-logo}
    \end{center}
    \vfill
    \begin{flushleft}
        \Huge
        \textbf{Git}
    \end{flushleft}
\end{frame}

\subsection{Kurzer Überblick über die Historie}\label{subsec:kurzer-uberblick-uber-die-historie}

\begin{frame}
    \slidehead
    \vspace{-1em}
    \begin{itemize}[<+->]
        \item 1991--2002 wurden Änderungen am Linux Kernel in Form von Patches herumgereicht
        \item Ab 2002 mit DVCS Bitkeeper (\enquote{Distributed Version Control System})
        \item 2005 ging die Beziehung \enquote{in die Brüche}
            \begin{itemize}
                \item Die zuvor ausgesprochene Erlaubnis, BitKeeper kostenlos zu verwenden, wurde widerrufen
            \end{itemize}
        \item April 2005 - Linus Torvalds fängt an, Git zu entwickeln.\ Die Ziele waren unter Anderem:
            \begin{itemize}
                \item Geschwindigkeit
                \item Einfaches Design
                \item Gute Unterstützung von nicht-linearer Entwicklung (tausende parallele Entwicklungszweige)
                \item Vollständig dezentrale Struktur
                \item Fähigkeit, große Projekte, wie den Linux Kernel, effektiv zu verwalten (Geschwindigkeit und Datenumfang)
            \end{itemize}
    \end{itemize}
    \renewcommand{\thefootnote}{\relax}\footnotetext{https://git-scm.com/book/de/v2/Erste-Schritte-Kurzer-\%C3\%9Cberblick-\%C3\%BCber-die-Historie-von-Git}
\end{frame}
