\newcommand{\mytree}{
    \begin{forest}
        for tree={anchor=center}
        [\faFolder, s sep=1em
        [\faFolder[\faFolder][\faFolder][\faFile*]]
        [\faFolder, before computing xy={s/.average={s}{siblings}}[\faFolder][\faFile*]]
        [\faFolder[\faFile*][\faFile*][\faFile*]]
        ]
    \end{forest}
}


\section{Dateisysteme}\label{sec:dateisysteme}

\begin{frame}[b]
    \begin{center}
        \includesvg[width = 8em]{../pictures/folder-tree-icon}
    \end{center}
    \vfill
    \begin{flushleft}
        \Huge
        \textbf{Dateisysteme}
    \end{flushleft}
\end{frame}

\begin{frame}[c]
    \centering
    \tikzmark{dolphin}\includesvg[width = 5em]{../pictures/dolphin-logo}
    \hspace{3em}
    \includesvg[width = 5em]{../pictures/gnome-files-logo}\tikzmark{nautilus}

    \vspace{3em}

    \tikzmark{finder}\includesvg[width = 5em]{../pictures/finder-logo}
    \hspace{3em}
    \includesvg[width = 5em]{../pictures/explorer-logo}\tikzmark{explorer}

    \pause
    \begin{tikzpicture}[remember picture, overlay]
        \node[anchor=west] at ([yshift=4ex]pic cs:dolphin) [xshift=-4cm, align=right, text width=3cm]{\Large\textbf{Dolphin}};
        \node[anchor=east] at ([yshift=4ex]pic cs:nautilus) [xshift=4cm, align=left, text width=3cm]{\Large\textbf{Nautilus}};
        \node[anchor=west] at ([yshift=4ex]pic cs:finder) [xshift=-4cm, align=right, text width=3cm]{\Large\textbf{Finder}};
        \node[anchor=east] at ([yshift=4ex]pic cs:explorer) [xshift=4cm, align=left, text width=3cm]{\Large\textbf{Explorer}};
    \end{tikzpicture}
\end{frame}

\subsection{Was ist ein Dateisystem?}\label{subsec:was-ist-ein-dateisystem}

\begin{frame}
    \slidehead
    \begin{center}
        \includesvg[width = 2em]{../pictures/dolphin-logo}
        \hspace{2em}
        \includesvg[width = 2em]{../pictures/gnome-files-logo}
        \hspace{2em}
        \includesvg[width = 2em]{../pictures/finder-logo}
        \hspace{2em}
        \includesvg[width = 2em]{../pictures/explorer-logo}

        \vspace{1em}
        Was haben diese Programme gemeinsam?
    \end{center}

    \vspace{1ex}
    \pause
    \begin{itemize}
        [<+->]
        \item Programme, um Dateien zu verwalten
        \item Diese Dateien befinden sich auf einem Dateisystem
        \begin{itemize}
            \item NTFS
            \item exFAT
            \item ext4
            \item APFS
            \item ZFS
            \item \dots und viele mehr
        \end{itemize}
    \end{itemize}
\end{frame}

\begin{frame}
    \slidehead
    Grundidee: Baumstruktur
    \\
    \vspace{1em}
    \begin{center}
        \begin{tikzpicture}
            \node[scale=1.7] at (4,4) {\mytree};
        \end{tikzpicture}
    \end{center}
\end{frame}

\subsection{Zusammenarbeit}\label{subsec:zusammenarbeit}

\begin{frame}
    \slidehead
    \begin{itemize}[<+->]
        \item Dateisysteme sind oberflächlich sehr ähnlich
        \begin{itemize}
            \item Ordner
            \item Dateien
        \end{itemize}
        \item Es gibt auch sehr viele Unterschiede
        \begin{itemize}
            \item OS Kompatibilität
            \item Line-Endings
            \item Metadaten
        \end{itemize}
        \item Für die Zusammenarbeit besonders wichtig:
        \begin{itemize}
            \item Wie können wir ein Dateisystem verteilen?
            \item Wie können wir den Zustand an einem bestimmten Zeitpunkt wiederherstellen?
        \end{itemize}
    \end{itemize}
\end{frame}

\subsection{Replikation}\label{subsec:replikation}

\begin{frame}[c]
    \slidehead
    \centering
    \Large
    Problem 1: Replikation
\end{frame}

\begin{frame}
    \slidehead
    Idee: Fileserver
    \vspace{-3em}
    \begin{center}
        \begin{tikzpicture}
            \node<1->[scale=0.5,draw,label={[anchor=south,font=\small]above:Globaler Zustand}] (server) at (0,0) {\mytree};
            \node<2->[scale=0.5,gray,opacity=0.5] (user-0) at (-6,-2.5) {\mytree};
            \node<2>[scale=0.5,gray,opacity=0.5,opacity=0] (user-fake) at (6,-2.5) {\mytree};
            \node<3->[scale=0.5,gray,opacity=0.5] (user-1) at (-2,-2.5) {\mytree};
            \node<3->[scale=0.5,gray,opacity=0.5] (user-2) at (2,-2.5) {\mytree};
            \node<3->[scale=0.5,gray,opacity=0.5] (user-3) at (6,-2.5) {\mytree};

            \node<2->[below=1ex of user-0,label={[anchor=north,font=\small]below:Du},orange!80] {\faUser};
            \node<3->[below=1ex of user-1,label={[anchor=north,font=\small]below:Mitglied \#1},blue!50,opacity=0.8] {\faUser};
            \node<3->[below=1ex of user-2,label={[anchor=north,font=\small]below:Mitglied \#2},blue!50,opacity=0.8] {\faUser};
            \node<3->[below=1ex of user-3,label={[anchor=north,font=\small]below:Mitglied \#3},blue!50,opacity=0.8] {\faUser};

            \draw<2-> ([yshift=-0.5ex,xshift=-1.3ex]server.south) -- (user-0.north);
            \draw<3-> ([yshift=-0.5ex,xshift=-0.5ex]server.south) -- (user-1.north);
            \draw<3-> ([yshift=-0.5ex,xshift=0.5ex]server.south) -- (user-2.north);
            \draw<3-> ([yshift=-0.5ex,xshift=1.3ex]server.south) -- (user-3.north);
        \end{tikzpicture}
    \end{center}
\end{frame}

\begin{frame}
    \slidehead
    \vspace{-1em}
    Idee: Fileserver
    \begin{itemize}
        [<+->]
        \item Ein Fileserver agiert als zentrale Schnittstelle und \enquote{Single source of truth}
        \begin{itemize}
            \item Writes werden direkt im Globalen Zustand geschrieben und sind für alle direkt sichtbar
            \item Reads liefern immer den aktuellsten Zustand
        \end{itemize}
        \item \enquote{Single point of failure} (SPOF)
        \item Keine klare \enquote{history}
        \begin{itemize}
            \item Veränderungen werden nicht gruppiert
            \item Unmöglich\footnote{Ja, es gibt Backups - dazu kommen wir noch} auf einen alten Zustand zurückzugehen
        \end{itemize}
        \item Unabhängige Arbeit unmöglich!
        \begin{itemize}
            \item Wenn gerade am Code gearbeitet wird, kompiliert er meistens nicht
            \item Dateien werden gegenseitig überschrieben
        \end{itemize}
    \end{itemize}
\end{frame}

\begin{frame}[c]
    \slidehead
    \Large
    \centering
    \enquote{Aber wir machen Backups!}
    \pause
    \\
    \vspace{1em}
    \textcolor{red!90}{Nein.}
\end{frame}

\subsection{Snapshots}\label{subsec:snapshots}

\begin{frame}[c]
    \slidehead
    \centering
    \Large
    Problem 2: Snapshots
\end{frame}

\begin{frame}
    \vspace{-2.5em}
    \slidehead
    \begin{center}
        \begin{tikzpicture}
            \tikzset{
                serverFeature/.style={
                    draw,
                    rounded corners=1ex,
                    minimum height=1.5em,
                    minimum width=3cm,
                },
                userFeature/.style={
                    draw,
                    rounded corners=1em,
                    inner sep=1em,
                },
            }

            \node[scale=0.5,draw,label={[anchor=south,font=\small]above:Globaler Zustand}] (server) at (0,0) {\mytree};
            \node[scale=0.5,gray,opacity=0.2] (user-0) at (-6,-2.5) {\mytree};
            \node[scale=0.5,gray,opacity=0.2] (user-1) at (-2,-2.5) {\mytree};
            \node[scale=0.5,gray,opacity=0.2] (user-2) at (2,-2.5) {\mytree};
            \node[scale=0.5,gray,opacity=0.2] (user-3) at (6,-2.5) {\mytree};

            \node[below=1ex of user-0,label={[anchor=north,font=\small]below:Du},orange!80] {\faUser};
            \node[below=1ex of user-1,label={[anchor=north,font=\small]below:Mitglied \#1},blue!50,opacity=0.8] {\faUser};
            \node[below=1ex of user-2,label={[anchor=north,font=\small]below:Mitglied \#2},blue!50,opacity=0.8] {\faUser};
            \node[below=1ex of user-3,label={[anchor=north,font=\small]below:Mitglied \#3},blue!50,opacity=0.8] {\faUser};

            \draw ([yshift=-0.5ex,xshift=-1.3ex]server.south) -- (user-0.north);
            \draw ([yshift=-0.5ex,xshift=-0.5ex]server.south) -- (user-1.north);
            \draw ([yshift=-0.5ex,xshift=0.5ex]server.south) -- (user-2.north);
            \draw ([yshift=-0.5ex,xshift=1.3ex]server.south) -- (user-3.north);

            \node[serverFeature,brown,above left=1ex and 6em of server] (server-backup-n) {Backup};

            \node<2>[userFeature,red] (feature-a) at (user-0) {Feature A};
            \node<2-3>[userFeature,blue!50!green] (fix-b) at (user-2) {Fix B};
            \node<2-4>[userFeature,red] (feature-C) at (user-3) {Feature C};

            \node<3->[serverFeature,red,below=1ex of server-backup-n] (server-feature-a) {Feature A};
            \node<4->[serverFeature,blue!50!green,below=1ex of server-feature-a] (server-fix-b) {Fix B};
            \node<5->[serverFeature,red,below=1ex of server-fix-b] (server-feature-c) {Feature C};

            \node<6>[right=2em of server,align=center,text width=5cm,orange,draw,label={[anchor=south,font=\small,align=left,text width=5cm]above:Frage:}] (problem) {Fix B rückgängig machen?};
        \end{tikzpicture}
    \end{center}
\end{frame}

\begin{frame}
    \slidehead
    \begin{itemize}
        [<+->]
        \item Projektzustand nur zu bestimmten Zeitpunkten gespeichert.
        \begin{itemize}
            \item Wiederherstellen von einzelnen Veränderungen sehr schwierig
            \item Vergleich von Zuständen sehr schwierig
            \item Kein \enquote{Blame}
            \item Backups sind keine Versionsverwaltung!
        \end{itemize}
        \item USB-Stick Verfahren ist noch ungeeigneter
    \end{itemize}
\end{frame}
