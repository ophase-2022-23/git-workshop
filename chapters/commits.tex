\section{Commits}\label{sec:commits}

\begin{frame}[b]
    \begin{center}
        \begin{tikzpicture}
            \node[commit] (commit) {c1c1c};
            \node[commit,left=2em of commit,fill opacity=0.4,text=black] (before) {$\cdots$};
            \node[commit,right=2em of commit,fill opacity=0.4,text=black] (after) {$\cdots$};

            \draw[parent] (commit.west) to (before.east);
            \draw[parent] (after.west) to (commit.east);
        \end{tikzpicture}
    \end{center}
    \vfill
    \begin{flushleft}
        \Huge
        \textbf{Commits}
    \end{flushleft}
\end{frame}

\begin{frame}
    \slidehead
    \begin{itemize}
        \item Ein Commit ist ein logischer Snapshot von einer Repository
        \begin{itemize}
            \item Zeitpunkt
            \item Dateisystemzustand
        \end{itemize}
        \item Datei hinzugefügt, geändert, verschoben oder gelöscht
        \item mit Nachricht versehen
        \item \enquote{kryptische} Bezeichnung (Hash), z.B. \texttt{bc7f1a9e22bc7f19e22bc}\dots
        \begin{itemize}
            \item erste fünf Zeichen zur Identifizierung meist ausreichend
        \end{itemize}
    \end{itemize}
\end{frame}

\begin{frame}[fragile]
    \slidehead
    Repository initialisieren
    \begin{columns}[c]
        \begin{column}{.5\textwidth}
            \centering
            \bashcommand{git init}
        \end{column}
        \begin{column}{.5\textwidth}
            \centering
            % @formatter:off
            \begin{codeBlock}[]{}
                .
                |-> .git
                    |-> HEAD
                    |-> config
                    |-> description
                    |-> hooks
                    |-> info
                    |-> objects
                    |-> refs
            \end{codeBlock}
            % @formatter:on
        \end{column}
    \end{columns}
\end{frame}

\subsection{Basics}\label{subsec:basics}

\begin{frame}
    \slidehead
    \vspace{1em}
    \centering
    \begin{tikzpicture}
        % state 1
        \node[commit](c1){c1c1c};
        \node<1>[head, below=2 em of c1](head0){};
        \draw<1>[parent] (head0) to (c1);
        % state 2
        \node<2->[commit, right=2 em of c1](c2){c2c2c};
        \draw<2->[parent] (c2) to (c1);
        \node<2>[head, below=2 em of c2](head1){};
        \draw<2>[parent] (head1) to (c2);
        % state 3
        \node<3->[commit, right=2 em of c2](c3){c3c3c};
        \draw<3->[parent] (c3) to (c2);
        \node<3>[head, below=2 em of c3](head2){};
        \draw<3>[parent] (head2) to (c3);
        % state 4
        \node<4->[branch, below=2em of c3](branch_master){master};
        \draw<4->[parent] (branch_master) to (c3);
        \node<4>[head, below=2em of branch_master](head2){};
        \draw<4>[parent] (head2) to (branch_master);
        % state 5
        \node<5->[head, below=2em of c2](head2){};
        \draw<5->[parent] (head2) to (c2);
    \end{tikzpicture}
    \vspace{1em}\par
    \only<1>{Initialer Commit \texttt{c1c1c}, z.B. \texttt{\enquote{Initial Commit}}}
    \only<2>{Commit \texttt{c2c2c}, z.B. \texttt{\enquote{Implement add and mul}}}
    \only<3>{Commit \texttt{c3c3c}, z.B. \texttt{\enquote{Fix sum}}}
    \only<4>{\texttt{HEAD} zeigt \textit{indirekt} auf einen Commit (hier \texttt{c3c3c})}
    \only<5>{\texttt{HEAD} kann auch \textit{direkt} auf Commit (hier \texttt{c2c2c}) zeigen}
\end{frame}

\subsection{Exkurs: Hashing}\label{subsec:exkurs:-hashing}

\begin{frame}[c]
    \slidehead
    \Huge
    \centering
    Was ist \enquote{Hashing}?
\end{frame}

\begin{frame}[c]
    \slidehead
    \centering
    {
        \Large
        \begin{tikzpicture}
            \node (file) {\fontsize{42pt}{42pt} \faFile*};
            \node[right=4em of file] (hash) {648a6a6ff...};

            \draw[-LaTeX, line width=2mm] (file) -- (hash);
        \end{tikzpicture}
    }
    \begin{align*}
        \mathbb{B} &:= \{0, 1\}^* \quad \text{(binäre Zahl, beliebige Länge)} \\
        \mathbb{B}_n &:= \{0, 1\}^n \quad \text{(binäre Zahl, Länge $n$)} \\
        h &: \mathbb{B} \rightarrow \mathbb{B}_n \\
        h(x) &= y \quad \forall x \in \mathbb{B}, \, \exists y \in \mathbb{B}_n
    \end{align*}
\end{frame}

\begin{frame}[c, fragile]
    \slidehead
    \centering
    \bashcommand{sha1sum file.txt}
    % @formatter:off
    \begin{codeBlock}[]{}
        648a6a6ffffdaa0badb23b8baf90b6168dd16b3a  file.txt
    \end{codeBlock}
    % @formatter:on
\end{frame}

\begin{frame}[c]
    \slidehead
    \centering
    {
        \Large
        \begin{tikzpicture}
            \node[scale=0.4] (file) {\mytree};
            \node[right=4em of file] (hash) {a72da862f...};

            \draw[-LaTeX, line width=2mm] (file) -- (hash);
        \end{tikzpicture}
    }
\end{frame}
