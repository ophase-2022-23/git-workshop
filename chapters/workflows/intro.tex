%! suppress = MissingImport
\section{Git Workflows}\label{sec:git-workflows}
\begin{frame}[c]
    \centering
    \Large
    \textbf{Git Workflows}
    \vspace{2em}
    \linebreak
    \begin{tikzpicture}
        \node[commit] at (0, 0)(a){A1A1A1};
        \node[commit] at (4, 0) (b){B2B2B2};
        \node[commit] at (4, -2) (c){C3C3C3};
        \node[branch] at (8, 0) (master){master};
        \node[branch] at (8, -2) (feature){feature};
        \draw[parent] (b) to (a);
        \draw[parent] (c.west) to (a.east);
        \draw[ref-arc] (master) to (b);
        \draw[ref-arc] (feature) to (c);
        \draw[parent, double] (feature) to (master);
    \end{tikzpicture}
\end{frame}

\begin{frame}[c]
    \slidehead
    \vspace{-1em}
    \centering
    \large
    \textbf{Was ist ein Git Workflow?}
    \vspace{2em}
    \only<2->{
        \linebreak
        \textit{Ein Git Workflow ist eine Rezeptur oder Empfehlung zur Verwendung von Git,
            die eine konsistente und produktive Arbeitsweise ermöglichen soll}
        \vspace{2em}
        \linebreak
        \textbf{
            \href{https://www.atlassian.com/de/git/tutorials/comparing-workflows}{Atlassian}
        }
    }
\end{frame}
