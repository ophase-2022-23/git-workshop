\documentclass[
    titleprefix=Git\space{}Cheat\space{}Sheet,
    inlineshortcut=java,
    corporatedesign,
    boxarc,
    % dark_mode,
]{algoexercise}

%%------------%%
%%--Packages--%%
%%------------%%

% \usepackage{audutils}
% \usepackage{fopbot}

%%---------------------------%%
%%--Dokumenteneinstellungen--%%
%%---------------------------%%

\subtitle{Alex und Dustin}
\dozent{Alex und Dustin} % chktex 12
\fachbereich{Informatik}
\semester{Ofahrt 2022}
\sheetnumber{0}
\title[Ausgabe von Listen]{}
\version{1.0}


\graphicspath{{../pictures/}}

%%----------------------------%%
%%--Stilistische Anpassungen--%%
%%----------------------------%%

% \ConfigureHeadline{
%     headline={aud-min}
% }

\renewcommand{\texttt}[1]{{\small{\textcolor{TUDa-3d}{\ttfamily\detokenize{#1}}}}}

\begin{document}%

    \small
    \section*{Git Cheat Sheet}

    % TODO anstatt commit auch branch

    \begin{minipage}[t]{.475\textwidth}
        \subsection*{GitLab und Moodle}
        Aus dem \textit{Myzel}-Netzwerk erreichst du folgende Seiten:
        \begin{itemize}
            \item GitLab | \href{http://gitlab.ofahrt2022.de/}{http://gitlab.ofahrt2022.de/}
            \item Moodle | \href{http://moodle.ofahrt2022.de/course/view.php?id=4} {http://moodle.ofahrt2022.de/}
        \end{itemize}

        \subsection*{Anmerkungen}

        Anstatt eines Commits kann auch ein Branch angegeben werden. Ein Branch refenziert einen Commit.

        \subsection*{Vorbereitung}
        Ein Repository, mit welchem du die nachfolgenden Operationen ausführen kannst,
        erhälst du über \textsf{init} oder \textsf{clone}.
        \begin{itemize}
            \item \texttt{git init} | erstellt lokales Repository im aktuellen Verzeichnis;
                Remote muss ggf. gesetzt werden
            \item \texttt{git clone <URL>} | cloned Remote-Repository in aktuelles Verzeichnis;
                Remote muss ist automatisch gesetzt
        \end{itemize}

        Zur Zuordnung deiner Commits solltest du deinen Namen und deine E-Mail-Adresse angeben.

        \begin{itemize}
            \item \texttt{git config --global user.name "<name>"} | setzt global den Namen
            \item \texttt{git config --global user.email "<mail>"} | setzt global die E-Mail-Adresse
        \end{itemize}


        \begin{itemize}
            \item \texttt{git push --set-upstream origin <branch>} |
                setzt \textsf{origin/<branch>} als Remote für lokales Repository

        \end{itemize}
        \subsection*{Staging}
        \begin{itemize}
            \item \texttt{git status} | zeigt Änderungen an
            \item \texttt{git add <files>} | \textit{staged} gegebene Dateien
            \item \texttt{git commit -m "<message>"} | \textit{commited} \textit{staged} Dateien mit gegebener Nachricht
            \item \texttt{git rm <files>} | \textit{staged} Lösch-Operation und entfernt Datei
            \item \texttt{git reset <files>} | \textit{unstaged} gegebene Dateien;
                Änderungen bleiben erhalten
        \end{itemize}
        \subsection*{Branching und Rebasing}
        \begin{itemize}
            \item \texttt{git branch} | listet Branches auf;
                aktueller Branch: \textsf{*}
            \item \texttt{git branch <name>} | leitet Branch vom aktuellen Branch mit gegebenem Namen ab
            \item \texttt{git checkout <commit>} | wechselt zu Commit
            \item \texttt{git checkout <branch>} | merged Branch in aktuellen
        \end{itemize}
    \end{minipage}%
    \begin{minipage}[t]{.025\textwidth}
        \hfill
    \end{minipage}%
    \begin{minipage}[t]{.475\textwidth}
        \subsection*{Zusammenarbeit}
        \begin{itemize}

            \item \texttt{git push} |
                schickt lokale Commits an Remote
            \item \texttt{git fetch} | lädt Änderungen von Remote,
                \textit{ohne} diese mit lokalen Branches zu mergen
            \item \texttt{git pull} | lädt Änderungen von Remote
                \textit{und} merged Änderungen für aktuellen Branch
        \end{itemize}

        \subsection*{Stashing}

        Änderungen, die \textit{noch} nicht commited werden sollen, können auf dem Stash-Stack zwischengespeichert und danach (auch auf anderem Branch) angewendet werden.

        \begin{itemize}
            \item \texttt{git stash} | legt Changes auf Stack
            \item \texttt{git stash list} | listet Stack auf
            \item \texttt{git stash pop} | wendet letzte Changes auf Stash an
        \end{itemize}

        \subsection*{Weiteres}
        \begin{itemize}
            \item \texttt{git remote add <name> <URL>} | fügt Remote unter gegebenem Namen hinzu;
                Standard-Name ist \texttt{origin};
                nicht notwendig bei \texttt{git clone}
            \item \texttt{git remove -v}\\Anzeigen der Remotes
        \end{itemize}
    \end{minipage}



\end{document}
