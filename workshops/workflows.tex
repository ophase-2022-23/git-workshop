%! suppress = FileNotFound
%! suppress = MissingImport
\RequirePackage{import}
\subimport{../common}{preamble}
\subimport{../common}{packages}
\subimport{../common}{vars}
\begin{document}
    \begin{frame}[c]
        \centering
        \Large
        \textbf{Effektive Git Workflows, Teamarbeit \& Best Practices}
    \end{frame}


    \section{Intro}\label{sec:intro}
    \begin{frame}[c]
        \slidehead
        \vspace{-1em}
        \centering
        \large
        \textbf{Das Heutige Programm}
        \vspace{1em}
        \begin{itemize}[<+->]
            \item Quick Recap
            \item Git Workflows \& Practices
                \begin{itemize}
                    \item Centralized Workflow
                    \item Feature Branch Workflow
                    \item GitHub Flow
                    \item Kurzer Einblick in andere Workflows
                \end{itemize}
            \item Git Platforms
            \item CI/CD
        \end{itemize}
    \end{frame}

    \subsection{Git Recap}\label{subsec:git-recap}
    \begin{frame}[c]
        \slidehead
        \vspace{-1em}
        \centering
        \large
        \textbf{Das Git-VCS}
        \vspace{1em}
        \begin{itemize}[<+->]
            \item Dezentrales Version Control System
            \item Ein \textbf{Commit} ist ein Zeiger auf einen Zustand vom Repository
            \item Eine \textbf{Branch} ist ein Zeiger auf einen \textbf{Commit}
        \end{itemize}
    \end{frame}

    \begin{frame}[c]
        \slidehead
        \vspace{-1em}
        \centering
        \large
        \textbf{Git Operations}
        \vspace{2em}
        \begin{description}[<+->][labelwidth=\widthof{\bfseries The longest label}]
            \item [git merge] Kombiniert zwei Branches
            \item [git squash merge] Merged nur den letzten Zustand der Branch
            \item [git rebase] Fügt neue commits hinter den commits der aktuellen Branch
            \item [git cherry-pick] Kopiert einen commit von einer anderen Branch
            \item [git revert] Erstellt einen Commit der einen anderen Commit rückgängig macht
        \end{description}
    \end{frame}

    \subimport{../chapters/workflows}{intro}
    \subimport{../chapters/workflows}{centralized}
    \subimport{../chapters/workflows}{feature-branch}
    \subimport{../chapters/workflows}{github-flow}
    \subimport{../chapters/workflows}{git-flow}
    \subimport{../chapters}{platforms}
    \subimport{../chapters}{ci-cd}


    \section{Nützliche Links}\label{sec:links}
    \begin{frame}[c]
        \slidehead
        \vspace{-1em}
        \centering
        \large
        \textbf{Nützliche Links}
        \vspace{1em}
        \begin{itemize}[<+->]
            \item \href{https://git-scm.com/book/en/v2}{Git Book}
            \item \href{https://www.atlassian.com/de/git/tutorials}{Atlassian Git Tutorials}
            \item \href{https://www.atlassian.com/de/git/tutorials/comparing-workflows}{Atlassian Git Workflows}
            \item \href{https://docs.github.com/en/get-started/quickstart/github-flow}{GitHub Flow}
            \item \href{https://nvie.com/posts/a-successful-git-branching-model/}{Git Flow}
        \end{itemize}
    \end{frame}
\end{document}
